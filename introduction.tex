%%%%%%%%%%%%%%%%%%%%%%%%%% author.tex %%%%%%%%%%%%%%%%%%%%%%%%%
%
% sample root file for your contribution to an IFIP volume
% published at Springer
%
% Use this file as a template for your own input.
%
%%%%%%%%%%%%%%%%%%%%%%%% Springer-Verlag %%%%%%%%%%%%%%%%%%%%%%%%%%


% RECOMMENDED %%%%%%%%%%%%%%%%%%%%%%%%%%%%%%%%%%%%%%%%%%%%%%%%%%%
\documentclass[ifip]{svmult}

% choose options for [] as required from the list
% in the Reference Guide, Sect. 2.2
\usepackage{epsfig} 
\usepackage{makeidx}         % allows index generation
\usepackage{graphicx}        % standard LaTeX graphics tool
                             % when including figure files
\usepackage{multicol}        % used for the two-column index
\usepackage[bottom]{footmisc}% places footnotes at page bottom
\usepackage[utf8]{inputenc}
% etc.
% see the list of further useful packages
% in the Reference Guide, Sects. 2.3, 3.1-3.3

\makeindex             % used for the subject index
                       % please use the style sprmidx.sty with
                       % your makeindex program


%%%%%%%%%%%%%%%%%%%%%%%%%%%%%%%%%%%%%%%%%%%%%%%%%%%%%%%%%%%%%%%%%%%%%

\begin{document}

\title*{FLOSS education and computational thinking workshop @ OSS 2016}
% Use \titlerunning{Short Title} for an abbreviated version of
% your contribution title if the original one is too long
\author{Jesús Moreno-León\inst{1}\and
Terhi Kilamo\inst{2}
\and
Gregorio Robles\inst{3}}
% Use \authorrunning{Short Title} for an abbreviated version of
% your contribution title if the original one is too long
\institute{Programamos.es, Spain,
\texttt{jesus.moreno@programamos.es}
\and Tampere University of Technology, Finland, \texttt{terhi.kilamo@tut.fi}
\and Universidad Rey Juan Carlos, Spain, \texttt{grex@gsyc.urjc.es}}
%
% Use the package "url.sty" to avoid
% problems with special characters
% used in your e-mail or web address
%
\maketitle

\section*{Introduction}
The presence of FLOSS in education has not stopped growing in the last years. The trend has been clear both in K-12 and higher education. While using FLOSS can support teaching computer science and other disciplines, its benefits lie in teaching FLOSS itself as part of the curriculum. 

An example that can illustrate this situation is the teaching of computational thinking skills through computer programming, which is one of the latest trends in education - for instance, Finland has just added coding and computational thinking as part of the national core curriculum for primary education. This field has been globally addressed almost exclusively with FLOSS technologies, both by using FLOSS platforms and programming languages, such as Scratch or Alice, but also by including in the curriculum the social aspects of software development that characterize FLOSS movements, like sharing and contributing to the community. 

The purpose of this workshop is to bring together free software experts and educators to discuss challenges that we face in the educational world at present and and that we will face in the future and how they can be undertaken from a FLOSS perspective.

\section*{Topics of interest}
The topics of interest of this workshop include but are not limited to:
\begin{itemize}
 \item Teaching experiences with FLOSS/free content
 \item FLOSS in higher education 
 \item FLOSS in K-12
 \item FLOSS practices in education 
 \item FLOSS in the curriculum 
 \item Computational thinking teaching and FLOSS
\end{itemize}
\subsection*{Specific research questions}
Specific questions that are of special interest to this workshop are: 
\begin{itemize}
 \item Which FLOSS approaches have proven beneficial to education?
 \item What experiences do you have in collaboration with FLOSS communities in education contexts?
 \item How do you produce and share your educational materials? 
 \item What assessment and certification models did you apply? Why have those models been chosen? 
 \item What models for sustainability and revenue generation did work successfully? 
 \item What efforts are undertaken to come towards a compatible or standardised curriculum? 
 \item What indicators are measured to show strengths and weaknesses of the initiative? 
 \item How is copyright and licensing managed in your institution/initiative? Which (potential) impact does this policy have on sustainability? 
 \item How can curriculi be designed which foster the spirit of sharing? 
 \item What FLOSS technologies do you use to teach computational thinking?
 \item How do you promote FLOSS social aspects in your lessons?
\end{itemize}
\section*{Programme Committee}
\begin{itemize}
	\item Yasemin Allsop, University of Roehampton (United Kingdom)
	\item Alessandro Bogliolo, University of Urbino (Italy)
	\item Jordi Freixenet, University of Girona (Spain)
	\item Petri Ihantola, Tampere University of Technology (Finland)
	\item Oystein Imsen, Lær Kidsa Koding (Norway)
	\item Terhi Kilamo, Tampere University of Technology (Finland)
	\item Joek van Montfort (Netherlands)
	\item Jesús Moreno, Programamos.es (Spain)
	\item Eduard Muntaner, Inventors for Change (Spain)
	\item Peter Parnes, Luleå University of Technology (Sweden)
	\item Gregorio Robles, Universidad Rey Juan Carlos (Spain)
\end{itemize}
\section*{Sponsorship}
This workshop is sponsored in part by the Region of Madrid under project ``eMadrid - Investigación y Desarrollo de tecnologías para el e-learning en la Comunidad de Madrid'' (S2013/ICE-2715).

%
%
% BibTeX users please use
% \bibliographystyle{}
% \bibliography{}
%
% Non-BibTeX users please follow the syntax
% the syntax of "referenc.tex" for your own citations
%\input{referenc}
%%%%%%%%%%%%%%%%%%%%%%%%%%%%%%%%%%%%%%%%%%%%%%%%%%%%%%%%%%%%%%%%%%%%%%

%%%%%%%%%%%%%%%%%%%%%%%%%%%%%%%%%%%%%%%%%%%%%%%%%%%%%%%%%%%%%%%%%%%%%%

\printindex
\end{document}





