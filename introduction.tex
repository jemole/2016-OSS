\documentclass[a4paper,10pt]{article}
\usepackage[utf8]{inputenc}

%opening
\title{FLOSS education and computational thinking workshop @ OSS 2016}
\author{}
\date{}

\begin{document}

\maketitle

%\begin{abstract}
%\end{abstract}

\section*{Introduction}
The presence of FLOSS in education has not stopped growing in the last years both in K-12 and higher education, a trend that has benefited from using FLOSS to teach computer science and other disciplines, but also for teaching FLOSS as part of the curriculum. An example that can illustrate this situation is the teaching of computational thinking skills through computer programming, which is one of the latest trends in education. This field has been globally addressed almost exclusively with FLOSS technologies, both by using FLOSS platforms and programming languages, such as Scratch or Alice, but also by including in the curriculum the social aspects of software development that characterize FLOSS movements, like sharing and contributing to the community. The purpose of this workshop is to bring together free software experts and educators to discuss challenges that we face in the educational world at present and and that we will face in the future and how they can be undertaken from a FLOSS perspective.

\section*{Topics of interest}
The topics of interest of this workshop include but are not limited to:
\begin{itemize}
 \item Teaching experiences with FLOSS/free content
 \item FLOSS in higher education 
 \item FLOSS in K-12
 \item FLOSS practices in education 
 \item FLOSS in the curriculum 
 \item Computational thinking teaching and FLOSS
\end{itemize}
\subsection*{Specific research questions}
Specific questions that are of special interest to this workshop are: 
\begin{itemize}
 \item How do you produce and share your educational materials? 
 \item What assessment and certification models did you apply? Why have those models been chosen? 
 \item What models for sustainability and revenue generation did work successfully? 
 \item What efforts are undertaken to come towards a compatible or standardised curriculum? 
 \item What indicators are measured to show strengths and weaknesses of the initiative? 
 \item How is copyright and licensing managed in your institution/initiative? Which (potential) impact does this policy have on sustainability? 
 \item How can curriculi be designed which foster the spirit of sharing? 
 \item What FLOSS technologies do you use to teach computational thinking?
 \item How do you promote FLOSS social aspects in your lessons?
\end{itemize}

\end{document}
